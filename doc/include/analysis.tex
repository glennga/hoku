\section{Analysis}\label{sec:analysis}

\subsection{Feature Uniqueness}\label{subsec:featureUniquenessAnalysis}
$\epsilon_1, \epsilon_2, \ldots \epsilon_n$ are hyperparameters that must be defined prior to executing the algorithm.
The heuristic used to determine each term was a grid search, where:
\begin{equation}
    \label{eq:gridSearchSigma}
    \epsilon_n \ \in \ \{1.0e^{-1}, 1.0e^{-2}, \ldots 1 .0e^{-14}\}
\end{equation}
Each experiment was performed $14n$ times , and the largest $\epsilon$ that produced the most query hits are displayed
below.
These were used for all experiments in the results presented below.
\begin{align*}
    \text{Angle}&: \sigma_\theta &= 1.0 \times 10^{-4}\\
    \text{Dot Angle}&: \sigma_{\theta_{ic}} &= 1.0 \times 10^{-4}\\
    \text{Dot Angle}&: \sigma_{\theta_{jc}} &= 1.0 \times 10^{-4}\\
    \text{Dot Angle}&: \sigma_\phi &= 1.0 \times 10^{-4} \\
    \text{Spherical Triangle}&: \sigma_a &= 1.0 \times 10^{-4}\\
    \text{Spherical Triangle}&: \sigma_\imath &= 1.0 \times 10^{-4}\\
    \text{Planar Triangle / Composite}&: \sigma_a &= 1.0 \times 10^{-4} \\
    \text{Planar Triangle / Composite}&: \sigma_\imath &= 1.0 \times 10^{-4}\\
\end{align*}

Each

\subsection{Candidate Reduction}\label{subsec:candidateReductionAnalysis}

\subsection{Alignment Determination}\label{subsec:alignmentDeterminationAnalysis}