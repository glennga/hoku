\section{Attitude Determination}\label{sec:attitudeDetermination}
\begin{figure}
    \centering{
    \includegraphics[scale=0.23]{images/moving-coordinate-systems.PNG}
    \caption{
    Visual of two coordinate frames: the inertial frame $A$, and the body frame $B$~\cite{CoordinateSystem}.
    The point $x$ is observed with vector $x_A$ in the $A$ frame, but the same point $x$ is observed with vector
    $x_B$ in the $B$ frame.
    By aligning several observations in both frames, a spacecraft orientation $x_{AB}$ in the $B$ frame can be
    determined in the $A$ frame.
    } \label{figure:coordinateSystem}
    }
\end{figure}

Attitude determination is the process of finding one's orientation in space.
On Earth, "up" and "down" mean "up from the center of the Earth" and "down toward the center of the Earth".
The direction toward the Earth can be determined using the Earth's gravitational presence.
This makes attitude determination fairly straightforward, requiring no more than a 3-axis acceleration sensor.
These are present inside most (if not all) modern cell phones.

In space, distinguishing "up from Earth" from "down toward Earth" is different as the Earth's gravitational presence is
severely diminished.
This means that the acceleration sensors used here cannot be used to determine one's attitude in space, and
spacecraft must use other objects to build an attitude from.
The visual position of the Sun and the Earth's magnetic field are common choices for reference points, but the most
popular are the visual positions of various stars in the sky.

Once some known object is found by the sensor, these measurements exists in a \textit{body reference frame}.
This reference frame reveals the position of objects with respect to the sensor, but is not fixed to any point.
There exists a separate, fixed frame with which these known objects are recorded to prior to the start of the mission.
This is known as an \textit{inertial reference frame}~\cite{Tappe}.
~\autoref{figure:coordinateSystem} depicts the relation between both frames.
Given sensor measurements to a set of points ($X_A, X'_A, X''_A, \ldots$) and inertial frame measurements to the same
objects ($X_B, X'_B, X''_B, \ldots$), the goal of attitude determination involves finding the rotation $X_{AB}$ to
take all points in the sensor frame to the inertial frame (or the inverse).
In short, attitude determination involves taking inertial frame object observations in the body frame, and figuring out
how the spacecraft is oriented in this inertial frame.

From here, the problem becomes aligning vector observations in the body frame with another set of vectors in the
inertial frame.
This is an optimization problem known as \textit{Wahba's problem}~\cite{AttitudeEstimation}.
A popular and simple method to extract an attitude from two vectors in both the inertial and body frame is the
\textit{TRIAD method}.
For all instances where Wahba's problem was present, the TRIAD method was used to extract an orientation.

\subsection{Stellar Based Attitude Determination}\label{subsec:stellarBasedAttitudeDetermination}
Relative to our solar system, the majority of bright stars ($m < 6.0$, or visible from the Earth without a telescope)
\textit{do not visibly move}.
Most of these stars exist hundreds of lightyears away from the solar system.
Observing a tiny change in a star's position from Earth suggests a massive change in position relative to the star.
For many LEO missions, the time it takes for these star positions to change is longer than the length of mission itself.
For longer missions, only then will star dynamics have to be taken into account.
This paper address spacecraft missions where all stars are virtually static in position.

A basic star tracker is composed of a camera, a computer for determining orientation, and a link back to the main
computer.
Once the star tracker on a lost-in-space spacecraft takes a picture, three main pieces of information are known:
\begin{enumerate}
    \item An 2D image of the sky.
    \item Characteristics of the camera hardware (field-of-view, lens structure, \ldots).
    \item All cataloged stars and their 3D positions in the catalog (inertial) frame.
\end{enumerate}

After taking the picture, the pixel positions of potential stars in the image are determined.
This involves finding bright blobs in the image, and computing each blob's center of mass.
To align these stars with the ones in the catalog, the image must be projected into three-dimensional.
This is done using the items in (2) above, and can be a major source of error if these
characteristics are not accurate.
The result of this process are star vectors in the image (body) frame~\cite{Tappe}.

By running these image star vectors through a star identification method, a map between the image frame stars and
catalog frame stars is found.
For a given image star $b$ and a catalog star $r$, a map $a$ describes an association ($b, r$) between both stars.
This then reduces to Wahba's problem, and is solved using the TRIAD method to obtain a spacecraft attitude.