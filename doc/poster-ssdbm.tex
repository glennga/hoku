\documentclass{beamer}

\usepackage[orientation=landscape, size=a0, scale=1.29]{beamerposter} \usepackage{graphicx} \usepackage{booktabs} \usepackage{tikz} \usepackage{tikz-3dplot} \usepackage{pgfplots} \usepackage{amsmath} \usepackage{amssymb} \usepackage{amsfonts} \usepackage{tabularx}

\usetheme{confposter} 
\usefonttheme{serif}

\setbeamercolor{block title}{fg=ngreen,bg=white}
\setbeamercolor{block body}{fg=black,bg=white}
\setbeamercolor{block alerted title}{fg=white,bg=dblue!70}
\setbeamercolor{block alerted body}{fg=black,bg=dblue!10}

\newlength{\sepwid}
\newlength{\onecolwid}
\newlength{\twocolwid}
\newlength{\threecolwid}
\setlength{\paperwidth}{48in}
\setlength{\paperheight}{36in} 
\setlength{\sepwid}{0.024\paperwidth}
\setlength{\onecolwid}{0.22\paperwidth} 
\setlength{\twocolwid}{0.464\paperwidth} 
\setlength{\threecolwid}{0.708\paperwidth}
\setlength{\topmargin}{-0.5in}

\newcommand{\drawsmallgraph}[1]{
	\begin{tikzpicture}[scale=5,tdplot_main_coords]
	    \draw[thick,->,color=dblue!70,line width=0.1cm] (0,0,0) -- (0.7,0,0) node[anchor=north east]{$v_1$};
	    \draw[thick,->,color=dblue!70,line width=0.1cm] (0,0,0) -- (0,0.7,0) node[anchor=north west]{$v_2$};
	    \draw[thick,->,color=dblue!70,line width=0.1cm] (0,0,0) -- (0,0,0.7) node[anchor=south]{$v_3$};
	    
	    \coordinate (O) at (0, 0, 0);
	    \coordinate (I_1) at (0.7, 0.6, 0.3);
	    \coordinate (I_2) at (0.3, 0.3, 0.7);
	    \coordinate (I_3) at (0.3, 0.7, 0.4);
		
		#1 % Extra draw code goes here.
	\end{tikzpicture}
}

\setbeamertemplate{headline}{ % Add our UH logo.
	\leavevmode
	\begin{columns}
		\begin{column}{0.92\linewidth}
			\vskip1cm
			\centering
			\usebeamercolor{title in headline}{\color{jblue}\Huge{\textbf{\inserttitle}}\\[0.5ex]}
			\usebeamercolor{author in headline}{\color{fg}\Large{\insertauthor}\\[1ex]}
			\usebeamercolor{institute in headline}{\color{fg}\large{\insertinstitute}\\[1ex]}
			\vskip1cm
		\end{column}
		\begin{column}{0.08\linewidth}
			\includegraphics[width=9.5cm]{images/uh-logo.png}
		\end{column}
		\vspace{1cm}
	\end{columns}
	\vspace{0.5in}
	\hspace{0.5in}\begin{beamercolorbox}[wd=47in,colsep=0.15cm]{cboxb}\end{beamercolorbox}
	\vspace{0.1in}
}

\title{An Experimental Survey of Evaluation Strategies for Constellation Queries}
\author{Glenn Galvizo \& Dr. Lipyeow Lim}
\institute{Information and Computer Science Department, University of Hawaii at Manoa}
\date{July, 2019}

\begin{document}
	\addtobeamertemplate{block end}{}{\vspace*{2ex}}
	\addtobeamertemplate{block alerted end}{}{\vspace*{2ex}}
	\setlength{\belowcaptionskip}{2ex}
	\setlength\belowdisplayshortskip{2ex}

	\begin{frame}[t]
		\begin{columns}[t]
			\begin{column}{\sepwid}\end{column}
	
			\begin{column}{\onecolwid} % First column.
				\begin{alertblock}{Objectives}
					\begin{itemize}
						\setlength\itemsep{1cm}
						\item \parbox{\linewidth}{
							\emph{Development} of unified framework for describing constellation query processing strategies.
							}
						\item \parbox{\linewidth}{
							\emph{Empirical evaluation} of six constellation query processing algorithms.
							}
					\end{itemize}
				\end{alertblock}
				\vspace*{1.5cm}
				\begin{definition}[Constellation Query]
					\begin{enumerate}
						\setlength\itemsep{1cm}
						\item \parbox{\linewidth}{
							Given query points in image coordinate system.
						}
						\item \parbox{\linewidth}{
							Identify a set of matching query points in a database of known points, recorded in a standard coordinate system.
							}
						\item \parbox{\linewidth}{
							Transformation between image and standard coordinate systems is not known a priori.
							}
					\end{enumerate}
					\medskip
					\begin{figure}
						\centering
						\tdplotsetmaincoords{60}{110}

% Origin for the secondary coordinate system (body), in spherical coordinates of the inertial.
\pgfmathsetmacro{\rvec}{.8} \pgfmathsetmacro{\thetavec}{30} \pgfmathsetmacro{\phivec}{60}

\begin{tikzpicture}[scale=3,tdplot_main_coords]

    % Draw the inertial frame.
    \coordinate (O) at (0,0,0);
    \node[color=red, anchor=east] at (0,0,0) {$\kFrame$ Frame};
    \draw[thick,->] (0,0,0) -- (0.7,0,0) node[anchor=north east]{$\vv{u_1}$};
    \draw[thick,->] (0,0,0) -- (0,0.7,0) node[anchor=north west]{$\vv{u_2}$};
    \draw[thick,->] (0,0,0) -- (0,0,0.7) node[anchor=south]{$\vv{u_3}$};

    % Plot the origin body frame in the inertial frame, show this connection.
    \tdplotsetcoord{P}{\rvec}{\thetavec}{\phivec}
%    \draw[dashed] (O) -- (P) node[anchor=north east, xshift=-0.5cm, yshift=-0.5cm]{$A^\nicefrac{I}{C}$};

    % Setup the body frame (don't draw till the end).
    \tdplotsetrotatedcoords{\phivec}{\thetavec}{0}
    \tdplotsetrotatedcoordsorigin{(P)}

    % Draw the point itself.
    \draw[tdplot_rotated_coords] (.205,.205,.205) circle[radius=0.3pt,fill=gray];
    \node[tdplot_rotated_coords, anchor=west, xshift=1cm, color=blue] at (.2,.2,.2) {$\vv{I_j}$};

    % Draw the point in the body frame.
    \draw[-stealth,color=blue,tdplot_rotated_coords] (0,0,0) -- (.2,.2,.2);
%    \node[color=blue,tdplot_rotated_coords,fill=white,xshift=0.6cm] at (.1,.1,.1) {};
    \draw[dashed,color=blue,tdplot_rotated_coords] (0,0,0) -- (.2,.2,0);
    \draw[dashed,color=blue,tdplot_rotated_coords] (.2,.2,0) -- (.2,.2,.2);

    % Draw the point in the inertial frame.
    \coordinate (Q) at (0.4,0.767,0.91);
    \draw[-stealth,color=red] (O) -- (Q);
    \node[color=red,fill=white,xshift=-0.1cm,yshift=-0.1cm] at (0.2,0.3835,0.455) {$\vv{K_j}$};
    \draw[dashed, color=red] (O) -- (0.4,0.767,0);
    \draw[dashed, color=red] (Q) -- (0.4,0.767,0);

    % Draw the body frame.
    \node[color=blue,tdplot_rotated_coords,anchor=east,xshift=1.1cm, yshift=1.2cm] at (0,0,0) {$\iFrame$ Frame};
    \draw[thick,tdplot_rotated_coords,->] (0,0,0) -- (.5,0,0) node[anchor=east]{$\vv{v_1}$};
    \draw[thick,tdplot_rotated_coords,->] (0,0,0) -- (0,.5,0) node[anchor=west]{$\vv{v_2}$};
    \draw[thick,tdplot_rotated_coords,->] (0,0,0) -- (0,0,.5) node[anchor=west]{$\vv{v_3}$};

\end{tikzpicture}

						\parbox{\linewidth}{\caption{\ \ 
							Shows the two coordinate systems in which points are represented.
							The resultant of a constellation query describes several points in both systems.
						}}
					\end{figure}
				\end{definition}
				\vspace*{0.5cm}
				\begin{block}{Properties of Constellation Queries}
					\begin{enumerate}
						\setlength\itemsep{1cm}
						\item \parbox{\linewidth}{
							\underline{\emph{Subgraph Isomorphism}} -- Find some 1-to-1 mapping between points in two graphs (database, image).
							}
						\item \parbox{\linewidth}{
							\underline{\emph{Non Recursive}} -- Results from previous constellation queries are not used in the current query.
							}
						\item \parbox{\linewidth}{
							\underline{\emph{Indexed by Positional Statistic}} -- Utilize the position of points relative to each other to index point sets, not star brightness.
							}
						\item \parbox{\linewidth}{
							\underline{\emph{Errors \& Noise}} -- False positives, false negatives, and deviations in true position affect the query point set.
							}
						\item \parbox{\linewidth}{
							\underline{\emph{Usage with Spacecraft}} -- Images of stars taken onboard spacecraft determine orientation of craft itself.
							}
					\end{enumerate}
				\end{block}
			\end{column}
			
			\begin{column}{\twocolwid}
				\begin{columns}[t,totalwidth=\twocolwid]
					\begin{column}{\onecolwid}\vspace{-.6in} % Second column.
						\begin{block}{Unified Identification Framework}
							\begin{figure}
								\centering
								\usetikzlibrary{shapes.geometric, arrows}

% Style for process block.
\tikzstyle{process} = [rectangle, text width=8cm, minimum width=8cm, minimum height=6cm,text centered, draw=black, fill=norange!70, line width=0.06cm]

% Style for terminal block.
\tikzstyle{terminal} = [rectangle, text width=8cm, minimum width=8cm, minimum height=6cm,text centered, draw=black, fill=nred!50, line width=0.06cm]

% Style for decision block.
\tikzstyle{decision} = [diamond, text width=6cm, minimum width=10cm, minimum height=8cm,text centered, draw=black,fill=lgreen, inner sep=-12pt, line width=0.06cm]

% Style for line.
\tikzstyle{line} = [draw, -latex', line width=0.1cm, ->]

\begin{tikzpicture}[node distance=9cm, scale=0.89, transform shape]
	\node(getImage)[terminal]{Given Image Set, Reference Database};
	\node(pickQueryStars)[process, left of=getImage, xshift=-4cm]{Select Subset from Image Set};
	\node(searchCatalog)[process, below of=pickQueryStars] {Query Reference with Feature Predicate(s)};
	\node(confidentInCatalog)[decision, below of=searchCatalog] {Query Yields Sole Result?};
	\node(findMap)[process, below of=confidentInCatalog]{Map Image and Reference Subsets Together};
	\node(confidentInMap)[decision, below of=findMap] {Confident in Map?};
	\node(returnMap)[terminal, right of=confidentInMap, xshift=4cm] {Return Mapping, Image Subset};
	
	\node[below of=getImage, yshift=5cm, xshift=4cm,scale=1.1]{$1$};
	\node[below of=pickQueryStars, yshift=5cm, xshift=4cm,scale=1.1]{$2$};
	\node[below of=searchCatalog, yshift=5cm, xshift=4cm,scale=1.1]{$3$};
	\node[below of=confidentInCatalog, yshift=7cm, xshift=4cm,scale=1.1]{$4$};
	\node[below of=findMap, yshift=5cm, xshift=4cm,scale=1.1]{$5$};
	\node[below of=confidentInMap, yshift=7cm, xshift=4cm,scale=1.1]{$6$};
	\node[below of=returnMap, yshift=5cm, xshift=4cm,scale=1.1]{$7$};
	
	\draw[line](getImage.west) -- (pickQueryStars.east);
	\draw[line] (pickQueryStars) -- (searchCatalog);
    \draw[line] (searchCatalog) -- (confidentInCatalog);
    \draw[line] (confidentInCatalog) -- node[anchor=east, yshift=0.1cm]{Yes}(findMap);
    \draw[line] (findMap) -- (confidentInMap);
    \draw[line] (confidentInMap) -- node[xshift=-0.5cm, yshift=0.75cm]{Yes} (returnMap);
    
    \draw[line] (confidentInCatalog.west) -- ++(-4cm, 0cm) node[anchor=south, xshift=2cm]{No} |- (pickQueryStars.west);
    \draw[line] (confidentInMap.west) -- ++(-4cm, 0cm) node[anchor=south, xshift=2cm]{No} |- (pickQueryStars.west);    
\end{tikzpicture}

%\begin{tikzpicture}[node distance=1.3cm, scale=1.5, transform shape]
%    \node[scale=1](getImage)[terminal]{Given Image Set, Reference Database};
%    \node[scale=1](pickQueryStars) [process, left of=getImage, xshift =-2.2cm] {Select Subset from Image Set};
%    \node[scale=1](searchCatalog)[process, below of=pickQueryStars, yshift=-0.7cm] {Query Reference with Feature Predicate(s)};
%    \node[scale=1](confidentInCatalog)[decision, below of=searchCatalog, yshift=-1cm] {Query Yields Sole Result?};
%%    \node[scale=1](filterCandidates)[process, below of=confidentInCatalog, yshift=-0.4cm] {Select Candidate $R[1]$};
%%    \node[scale=1](confidentAfterFilter)[decision, below of=filterCandidates, yshift=-0.4cm] {Confident?};
%    \node[scale=1](findMap)[process, below of=confidentInCatalog, yshift=-1.2cm]{Map Image and Reference Subsets Together};
%    \node[scale=1](confidentInMap)[decision, below of=findMap, yshift=-1cm] {Confident in Map?};
%    \node[scale=1](returnMap)[terminal, right of=confidentInMap, xshift = 2.2cm] {Return Mapping, Image Subset};
%
%    \draw[->,>=stealth](getImage.west) -- (pickQueryStars.east);
%    \draw[->,>=stealth] (pickQueryStars) -- (searchCatalog);
%    \draw[->, >=stealth] (searchCatalog) -- (confidentInCatalog);
%    \draw[->,>=stealth] (confidentInCatalog) -- node[anchor=east, yshift=0.1cm]{Yes}(findMap);
%%    \draw[->, >=stealth] (confidentInCatalog) -- node[anchor=east, yshift=0.1cm]{Yes}(filterCandidates);
%%    \draw[->, >=stealth] (filterCandidates) -- (findMap);
%%    \draw[->, >=stealth] (confidentAfterFilter) -- node[anchor=east, yshift=0.1cm]{Yes} (findMap);
%    \draw[->, >=stealth] (findMap) -- (confidentInMap);
%    \draw[->, >=stealth] (confidentInMap) -- node[xshift=0cm, yshift=0.25cm]{Yes} (returnMap);
%
%    \draw[->, >=stealth] (confidentInCatalog.west) -- ++(-1.4cm, 0cm) node[anchor=south, xshift=0.5cm]{No}
%    |- (pickQueryStars.west);
%%    \draw[->, >=stealth] (confidentAfterFilter.west) -- ++(-1.4cm, 0cm) node[anchor=south, xshift=0.5cm]{No}
%%    |- (pickQueryStars.west);
%    \draw[->, >=stealth] (confidentInMap.west) -- ++(-1.4cm, 0cm) node[anchor=south, xshift=0.5cm]{No}
%    |- (pickQueryStars.west);
%\end{tikzpicture}
								\caption{
									Depicts framework which all strategies here follow.
									If all image subsets are exhausted, an error is raised and no mapping is returned (not depicted).
								}
							\end{figure}
						\end{block}
					\end{column}
					
					\begin{column}{\onecolwid}\vspace{-.6in} % Third column.
						\begin{block}{Methodology}
							\begin{itemize}
								\item 1{,}471 points from Hipparcos catalog based on brightness.
								\item Simulated o
							\end{itemize}
						\end{block}
				
						\begin{block}{Results}
							
							Include graphic for time 
					
							Include graphic for accuracy
						\end{block}
					\end{column}
				\end{columns}	
				
				\begin{block}{Strategy Overview \& Characteristics}
					\tdplotsetmaincoords{60}{110}
					\begin{table}
						\begin{tabularx}{\linewidth}[t]{X|X|X|X|X|X}
							\toprule
							{ANG} & {INT} & {SPH} & {PLN} & {PYR} & {COM} \\
							\midrule
							
							\drawsmallgraph{
								\draw[color=norange!70, line width=0.05cm](O) -- (I_2);
	    						\draw[color=norange!70, line width=0.05cm](O) -- (I_3);
	    
							    \filldraw[color=jblue] (I_2) circle[radius=0.5pt];					 
							    \filldraw[color=jblue] (I_3) circle[radius=0.5pt];
							    
							    \tdplotdefinepoints(0,0,0)(0.3,0.3,0.7)(0.3,0.7,0.4)
							    \tdplotdrawpolytopearc[color=nred!50,line width=0.05cm]{0.6}{yshift=0.5cm,xshift=0.5cm}{\textcolor{nred!50}{$\theta$}}
							} &
							\drawsmallgraph{
								\draw[color=norange!70, line width=0.05cm](O) -- (I_1);
								\draw[color=norange!70, line width=0.05cm](O) -- (I_2);
	    						\draw[color=norange!70, line width=0.05cm](O) -- (I_3);
	    						\draw[color=norange!70, line width=0.05cm, dashed](I_3) -- (I_2);
	    						\draw[color=norange!70, line width=0.05cm, dashed](I_2) -- (I_1);
								
								\tdplotdefinepoints(0,0,0)(0.3,0.3,0.7)(0.3,0.7,0.4)
							    \tdplotdrawpolytopearc[color=nred!50,line width=0.05cm]{0.6}{yshift=1.1cm,xshift=0.5cm}{\textcolor{nred!50}{$\theta_1$}}
							    	\tdplotdefinepoints(0,0,0)(0.3,0.3,0.7)(0.7,0.6,0.3)
							    \tdplotdrawpolytopearc[color=nred!50,line width=0.05cm]{0.6}{yshift=-1.5cm,xshift=-0.5cm}{\textcolor{nred!50}{$\theta_2$}}
							    	\tdplotdefinepoints(0.3,0.3,0.7)(0.3,0.7,0.4)(0.7,0.6,0.3)
							    \tdplotdrawpolytopearc[color=nred!50,line width=0.05cm]{0.4}{xshift=1.6cm}{\textcolor{nred!50}{$\phi$}}
							    
%								\coordinate (I_1) at (0.7, 0.6, 0.3);
%							    \coordinate (I_2) at (0.3, 0.3, 0.7);
%							    \coordinate (I_3) at (0.3, 0.7, 0.4);
								
								\filldraw[color=jblue] (I_1) circle[radius=0.5pt];
								\filldraw[color=jblue] (I_2) circle[radius=0.5pt];					 
							    \filldraw[color=jblue] (I_3) circle[radius=0.5pt];
							} &
							\drawsmallgraph{
								\path [bend left,color=norange!70, line width = 0.05cm] (I_2) edge (I_1);
								\path [bend left,color=norange!70, line width = 0.05cm] (I_2) edge (I_3);
								\path [bend left,color=norange!70, line width = 0.05cm] (I_3) edge (I_1);
								
								\draw[thick,->,color=nred!50,line width=0.05cm] (0.4,0.6,0.5) -- (0.8, 1.0, 0.9);
								\node[color=nred!50,anchor=west,xshift=0.2cm, yshift=0.2cm] at (I_2) {$a$};
								\node[color=nred!50,anchor=west] at (0.8, 1.0, 0.9) {$j$}; 
								
								\filldraw[color=jblue] (I_1) circle[radius=0.5pt];
								\filldraw[color=jblue] (I_2) circle[radius=0.5pt];					 
							    \filldraw[color=jblue] (I_3) circle[radius=0.5pt];
							    \filldraw[color=norange] (0.4, 0.6, 0.5) circle[radius=0.5pt];
							} &
							\drawsmallgraph{
								\draw[color=norange!70, line width = 0.05cm] (I_2) -- (I_1);
								\draw[color=norange!70, line width = 0.05cm] (I_2) -- (I_3);
								\draw[color=norange!70, line width = 0.05cm] (I_3) -- (I_1);
								
								\draw[thick,->,color=nred!50,line width=0.05cm] (0.4,0.55,0.5) -- (0.8, 1.0, 1.0);
								\node[color=nred!50,anchor=west,xshift=0.2cm, yshift=0.2cm] at (I_2) {$a$};
								\node[color=nred!50,anchor=west] at (0.8, 1.0, 01.0) {$j$}; 
								
								\filldraw[color=jblue] (I_1) circle[radius=0.5pt];
								\filldraw[color=jblue] (I_2) circle[radius=0.5pt];					 
							    \filldraw[color=jblue] (I_3) circle[radius=0.5pt];
							    \filldraw[color=norange] (0.4, 0.55, 0.5) circle[radius=0.5pt];
							} &
							\drawsmallgraph{
								\draw[color=norange!70, line width=0.05cm](O) -- (I_1);
								\draw[color=norange!70, line width=0.05cm](O) -- (I_2);
	    						\draw[color=norange!70, line width=0.05cm](O) -- (I_3);
	    						
								\tdplotdefinepoints(0,0,0)(0.3,0.3,0.7)(0.3,0.7,0.4)
							    \tdplotdrawpolytopearc[color=nred!50,line width=0.05cm]{0.6}{yshift=1.1cm,xshift=0.5cm}{\textcolor{nred!50}{$\theta_1$}}
							    	\tdplotdefinepoints(0,0,0)(0.3,0.3,0.7)(0.7,0.6,0.3)
							    \tdplotdrawpolytopearc[color=nred!50,line width=0.05cm]{0.6}{yshift=-1.5cm,xshift=-0.5cm}{\textcolor{nred!50}{$\theta_2$}}
							    	\tdplotdefinepoints(0,0,0)(0.3,0.7,0.4)(0.7,0.6,0.3)
							    \tdplotdrawpolytopearc[color=nred!50,line width=0.05cm]{0.6}{yshift=-1cm,xshift=1cm}{\textcolor{nred!50}{$\theta_3$}}
	    						
								\filldraw[color=jblue] (I_1) circle[radius=0.5pt];
								\filldraw[color=jblue] (I_2) circle[radius=0.5pt];					 
							    \filldraw[color=jblue] (I_3) circle[radius=0.5pt];
							} &
							\drawsmallgraph{
								\draw[color=norange!70, line width = 0.05cm] (I_2) -- (I_1);
								\draw[color=norange!70, line width = 0.05cm] (I_2) -- (I_3);
								\draw[color=norange!70, line width = 0.05cm] (I_3) -- (I_1);
								
								\draw[thick,->,color=nred!50,line width=0.05cm] (0.4,0.55,0.5) -- (0.8, 1.0, 1.0);
								\node[color=nred!50,anchor=west,xshift=0.2cm, yshift=0.2cm] at (I_2) {$a$};
								\node[color=nred!50,anchor=west] at (0.8, 1.0, 01.0) {$j$}; 
								
								\filldraw[color=jblue] (I_1) circle[radius=0.5pt];
								\filldraw[color=jblue] (I_2) circle[radius=0.5pt];					 
							    \filldraw[color=jblue] (I_3) circle[radius=0.5pt];
							    \filldraw[color=norange] (0.4, 0.55, 0.5) circle[radius=0.5pt];
							} \\
							
							${}^1$ Precompute and index point pairs by $\theta$. \\
							${}^3$ Search for database point pairs whose $\theta$ is close to the image subset $\theta$.
							
							&
							
							&
							
							&
							
							&
							
							&
							
							\\
							\bottomrule
						\end{tabularx}
					\end{table}
				\end{block}
			\end{column}
			
			\begin{column}{\onecolwid} % Fourth column.
				\begin{alertblock}{Conclusion}
					
				\end{alertblock}

				\begin{block}{Future Work}
					
				\end{block}
			
				\begin{block}{Acknowledgements}
					We would like to thank Dr. Miguel Nunes, Eric Pilger, and Yosef Ben Gershom from the Hawaii Space Flight Laboratory for providing input toward the creation of software for a first generation star tracker.
				\end{block}
			\end{column}
		\end{columns}				
	\end{frame}
\end{document}